\documentclass[12pt]{report}

% Encoding utf-8:
\usepackage[utf8]{inputenc} 

% Para separar sílabas e gerar texto em português do Brasil:
\usepackage[brazil]{babel}

% Para bibliografia em português
\usepackage{babelbib}

% Para usar fontes melhores:
\usepackage[T1]{fontenc}

% Para usar comandos matemáticos da AMS:
\usepackage{amssymb,amsmath,amsthm}

% Para poder usar comandos condicionais:
\usepackage{ifthen}

% Para gerar símbolos de conjuntos de números (\Nat, etc.):
\usepackage{bbm}

% \dotdiv
\usepackage{mathabx}

% Variable-size \Set and \mid
\usepackage{braket}

% Para incluir imagens:
\usepackage{graphicx}

% Para gerar listas com recursos adicionais:
\usepackage{paralist}

% Para formatar exercícios e soluções:
\usepackage{answers}

% Para títulos de capítulos e seções:
\usepackage{titlesec}

%%%%%%%%%%%%%%%%%%%%%%%%%%%%%%%%%%%%%%%%%%%%%%%%%%%%%%%%%%%%%%%%%%%%%%%%%%%
%
% Para a página de título:
%
%%%%%%%%%%%%%%% início do código comum da página de título %%%%%%%%%%%%%%%%
\usepackage{lmodern}
\usepackage{url}
\usepackage[svgnames]{xcolor}
%\ifpdf
\usepackage{pdfcolmk}
%\fi
%% check if using xelatex rather than pdflatex
%\ifxetex
%\usepackage{fontspec}
%\fi
\usepackage{graphicx}
%%\usepackage{hyperref}
%% drawing package
\usepackage{tikz}
%% for dingbats
\usepackage{pifont}

\providecommand{\HUGE}{\Huge}% if not using memoir
\newlength{\drop}% for my convenience
%% specify the Webomints family
\newcommand*{\wb}[2]{\fontsize{#1}{#2}\usefont{U}{webo}{xl}{n}}
%% select a (FontSite) font by its font family ID
\newcommand*{\FSfont}[1]{\fontencoding{T1}\fontfamily{#1}\selectfont}
%% if you don’t have the FontSite fonts either \renewcommand*{\FSfont}[1]{}
%% or use your own choice of family.
%% select a (TeX Font) font by its font family ID
\newcommand*{\TXfont}[1]{\fontencoding{T1}\fontfamily{#1}\selectfont}
%% Generic publisher’s logo
\newcommand*{\plogo}{\fbox{$\mathcal{PL}$}}

%% Some shades
\definecolor{Dark}{gray}{0.2}
\definecolor{MedDark}{gray}{0.4}
\definecolor{Medium}{gray}{0.6}
\definecolor{Light}{gray}{0.8}
%%%% Additional font series macros
\makeatletter
%%%% light series
%% e.g., kernel doc, section s: line 12 or thereabouts
\DeclareRobustCommand\ltseries
{\not@math@alphabet\ltseries\relax
\fontseries\ltdefault\selectfont}
%% e.g., kernel doc, section t: line 32 or thereabouts
\newcommand{\ltdefault}{l}
%% e.g., kernel doc, section v: line 19 or thereabouts
\DeclareTextFontCommand{\textlt}{\ltseries}
% heavy(bold) series
\DeclareRobustCommand\hbseries
{\not@math@alphabet\hbseries\relax
\fontseries\hbdefault\selectfont}
\newcommand{\hbdefault}{hb}
\DeclareTextFontCommand{\texthb}{\hbseries}
\makeatother

%%%%%%%%%%%%%%% fim do código comum da página de título %%%%%%%%%%%%%%%%

% titleBC %%%%%%%%%%%%%%%%%%%%%%%%%%%%%%%%%%%%%%%%%%%%%%%%%%%%%%%%%%%%%
%
\newcommand*{\rotrt}[1]{\rotatebox{90}{#1}}
\newcommand*{\rotlft}[1]{\rotatebox{-90}{#1}}
\newcommand*{\topb}{%
\resizebox*{1.5\unitlength}{\baselineskip}{\rotrt{$\}$}}}
\newcommand*{\botb}{%
\resizebox*{1.5\unitlength}{\baselineskip}{\rotlft{$\}$}}}
\newcommand*{\titleBC}{\begingroup
\FSfont{5jr}% Fontsite Jenson Recut (Centaur)
\begin{center}
\def\CP{\textit{\HUGE Lógica Matemática}}
\settowidth{\unitlength}{\CP}
{\color{LightGoldenrod}\topb} \\[\baselineskip]
\textcolor{Sienna}{\CP} \\[\baselineskip]
{\color{RosyBrown}\LARGE \&} \\[\baselineskip]
{\color{RosyBrown}\LARGE Técnicas de Demonstração} \\
{\color{LightGoldenrod}\botb}
\end{center}
\vfill
\begin{center}
{\LARGE\textbf{Fernando Náufel do Amaral}}\\
\vfill
Departamento de Ciências da Natureza\\
Pólo Universitário de Rio das Ostras\\
Universidade Federal Fluminense\\[0.5\baselineskip]
2018
\end{center}
\endgroup}

% titleGM
% titleCC
% titleGP
% titleBWF
% titleS
% titleTMB


% Para caixas:
\usepackage[breakable]{tcolorbox}

% Para hiperlinks:
\usepackage[colorlinks]{hyperref}


%%%%%%%%%%%%%%%%%%%%%%%%%%%%%%%%%%%%%%%%%%%%%%%%%%%%%
%
% Definições de novos comandos e ambientes:
%

% Símbolo do conjunto dos números naturais:
\newcommand{\Nat}{\ensuremath{\mathbbm{N}}}
\newcommand{\Int}{\ensuremath{\mathbbm{Z}}}
\newcommand{\Rac}{\ensuremath{\mathbbm{Q}}}
\newcommand{\Reais}{\ensuremath{\mathbbm{R}}}

% Símbolo de predicado (fonte sem serifa):
\newcommand{\pred}[1]{\textsf{#1}}

% Constantes (fonte sem serifa):
\newcommand{\const}[1]{\textsf{#1}}

% Várias constantes. Crie mais, se necessário.
% Use em modo matemático.
\newcommand{\ca}{\const{a}}
\newcommand{\cb}{\const{b}}
\newcommand{\cc}{\const{c}}
\newcommand{\cd}{\const{d}}
\newcommand{\ce}{\const{e}}

% Símbolos de função (fonte sem serifa, negrito):
\newcommand{\func}[1]{\textsf{\textbf{#1}}}

% Várias funções
\newcommand{\sen}{\func{sen}}
\newcommand{\arcsen}{\func{arcsen}}


%%%%%%%%%%%%%%%%%

% Exemplos numerados por seção:
\newtheoremstyle{exemplostyle}%
{4ex}% space above
{4ex}% space below
{}% body font
{}% indent
{\bfseries}% head font
{}% punctuation after head
{\newline}% space after head
{%
  \tcbox[colback=black!10!white,colframe=white,boxrule=.5pt]{%
    \thmname{#1}\thmnumber{ #2}\thmnote{ (#3)}%
  }%
}% head spec

\theoremstyle{exemplostyle}
\newtheorem{exemplo}{Exemplo}[section]

%%%%%%%%%%%%%%%%%%

% Solução de exemplo, sem numeração:
\newtheoremstyle{solucaostyle}%
{}% space above
{}% space below
{}% body font
{}% indent
{\bfseries}% head font
{\newline}% punctuation after head
{\newline}% space after head
{Solução:}% head spec

\theoremstyle{solucaostyle}
\newtheorem*{solucaoexemplo}{Solução}

%%%%%%%%%%%%%%%%%

% Soluções de exercícios
\Newassociation{sol}{Solution}{solucoes}

% Não é possível usar listas dentro do ambiente sol quando uso o
% ambiente list! O primeiro item fica sem label, e o label do primeiro
% item é impresso em cima do label do segundo item.
%
% FIXED: Redefini Solution para não usar ambiente list!

% Bugged:
% \renewenvironment{Solution}[1]
% {%
%   \begin{list}
%     {}
%     {%
%       \setlength{\leftmargin}{3em}%
%       \item[\textbf{#1)}\quad]%
%     }%
% }%
% {
%   \end{list}
%   \vskip 4ex
% }

\renewenvironment{Solution}[1]
{%
  \begin{description}
  \item[\textbf{#1)}]\mbox{}
    \vspace*{-2\parsep}
    \vspace*{-2\parskip}
}%
{
  \end{description}
}

\newenvironment{exercicios}[1]
{%
  \Writetofile{solucoes}{%
    \protect\section*{#1}%
%    \protect\addtocontents{toc}{#1}%
    \vspace{2ex}
  }

  \vspace{2ex}
  \begin{enumerate}[\bfseries 1.]
}
{%
  \end{enumerate}
  \Writetofile{solucoes}{%
    \protect\newpage
  }
}

%%%%%%%%%%%%%%%%%

% Destaque

\newcommand{\destaque}[2][]{%

  \vskip 2ex
  \begin{tcolorbox}[colback=black!5!white,colframe=black!70!white,boxrule=.5pt,#1]
    #2
  \end{tcolorbox}
  \vskip 2ex

}

% Implementação antiga, sem tcolorbox
%
% \newcommand{\destaque}[1]{%
%   \begin{center}
%     \fbox{%
%       \fbox{%
%           \parbox{.9\linewidth}{%
%             \begin{center}
%               #1
%             \end{center}
%           }%
%       }%
%     }
%   \end{center}
% }


%%%%%%%%%%%%%%%
%
% Titulos de capítulos e seções

\titleformat{\chapter}[display]
  {\bfseries\Large}
  {\filleft\MakeUppercase{\chaptertitlename} \Huge\thechapter}
  {4ex}
  {\titlerule
    \vspace{2ex}%
    \filright}
  [\vspace{2ex}%
  \titlerule
  \vspace{10ex}]

\titleformat{\section}[block]
  {\bfseries\Large}
  {\thesection}{.5em}{\titlerule\\[.8ex]\bfseries}

\titleformat{\subsection}[block]
  {\bfseries}
  {\thesubsection}{.5em}{\titlerule\\[.8ex]\bfseries}

%%% Local Variables: 
%%% mode: latex
%%% TeX-master: "main"
%%% End: 
